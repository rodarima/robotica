% Usar el tipo de documento: Artículo científico.
\documentclass[10pt,a4paper,hidelinks]{article}

% Cargar mensajes en español.
\usepackage[spanish,es-noquoting]{babel}

% Usar codificación utf-8 para acentos y otros.
\usepackage[utf8]{inputenc}
\usepackage[T1]{fontenc}
\usepackage{lmodern}

% Usar dos columnas
\usepackage{multicol}

% Separar las columnas
\setlength{\columnsep}{1cm}

% Dimensiones de los márgenes.
\usepackage[margin=1.8cm]{geometry}

% Insertar porciones de código
\usepackage{listings}

% Comenzar párrafos con separación no indentación.
\usepackage{parskip}
%enlaces
\usepackage{hyperref}
% Usar gráficos
\usepackage{graphicx}
\usepackage{caption}
\usepackage{subcaption}
%
% Usar contenedores flotantes para figuras.
\usepackage{float}

% Carpeta de las imágenes.
\graphicspath{{img/}}

% Matemáticas
\usepackage{amsmath}

% Dibujos geométricos
\usepackage{tikz}
\usetikzlibrary{through,calc,arrows,angles,babel}

\newcommand\markangle[6][red]{% [color] {X} {origin} {Y} {mark} {radius}
	% filled circle: red by default
	\begin{scope}
		\path[clip] (#2) -- (#3) -- (#4);
		\fill[color=#1,fill opacity=0.5,draw=#1,name path=circle]
		(#3) circle (#6mm);
	\end{scope}
	% middle calculation
		\path[name path=line one] (#3) -- (#2);
	\path[name path=line two] (#3) -- (#4);
	\path[%
		name intersections={of=line one and circle, by={inter one}},
		     name intersections={of=line two and circle, by={inter two}}
	] (inter one) -- (inter two) coordinate[pos=.5] (middle);
	% bissectrice definition
		\path[%
		name path=bissectrice
		] (#3) -- (barycentric cs:#3=-1,middle=1.2);
	% put mark
		\path[
		name intersections={of=bissectrice and circle, by={middleArc}}
		] (#3) -- (middleArc) node[pos=1.3] {#5};
}


\begin{document}

%\maketitle
\begin{center}
\begin{huge}
\textbf{Cucabot}
\end{huge}
\\[10pt]
\textbf{Rodrigo Arias Mallo}\\
rodrigo.arias@udc.es
\end{center}

\newcommand\RobotAngle{45}
\newcommand\RobotSize{2}
\newcommand\RobotRadius{4}
\newcommand\RobotThetaSonar{60.0}

\begin{multicols}{2}

\section{Descripción}
Cucabot es un robot diseñado para aprender los principios básicos de la robótica.
Ha sido creado con la intención de reutilizar componentes de antiguos aparatos y 
darles un nuevo uso.

\subsection{Sensores}
Cuenta con varios sensores que le permiten obtener información del entorno, así
como información sobre sí mismo. Son los siguientes:

\begin{enumerate}
	% Quitar espacio entre lineas
	\setlength{\parskip}{0cm}

	\item Sensor de ultrasonidos HC-SR04
	\item Ratón de bola
	\item Sensor de luz LDR dirigido hacia delante
	\item Sensor de luz LDR ambiental
\end{enumerate}

\subsection{Motores}
Para efectuar el movimiento, dispone de dos motores Mabuchi RF-500TB-12560, colocados
a ambos lados.

\subsection{Ratón}

En la parte posterior, un ratón de bola permite obtener mediciones del movimiento del
robot. Tiene un controlador EM83702BP, que dispone con dos contadores digitales que
realizan la medición del desplazamiento en el eje x e y.

Una bola de goma (caucho?) se encuentra en contacto con el suelo. Al desplazar el ratón, 
la bola gira alrededor de un eje imaginario, que es perpendicular al sentido del
desplazamiento y paralelo al suelo.

%\begin{center}
%\centering
%\includegraphics[scale=0.50]{mouse.pdf}
%\captionof{figure}{Bola del ratón y %los dos ejes. Medidas en mm.}
%\end{center}

Ya que los movimientos efectuados por el robot serán los que provoquen sus dos ruedas,
se puede determinar la posición, midiendo sólo el desplazamiento del ratón. Para ello,
se observa en la figura~\ref{fig:giro} un ejemplo de un giro de $\alpha = \RobotAngleº$.

\begin{center}
\begin{tikzpicture}[>=latex]
    %\path[draw] (-4,0)  coordinate [label= left:$A$] (A)
    %        -- ( 0,4)  coordinate [label=above:$C$] (C)
    %        -- ( 4,0)  coordinate [label=right:$B$] (B)
    %        -- cycle;
    %\foreach \point in {A,B,C}
    %       \fill [black] (\point) circle (2pt);
    %\draw [color=red] circle(4cm);
    
    
    %\draw (A) rectangle (1,1);
    
    %\def\robotsize{(1,1)}
    %\pgfmathsetmacro{\RobotSize}{2.0}%
    %\pgfmathsetmacro{\RobotAngle}{-45.0}%
    %\pgfmathsetmacro{\RobotRadius}{6.0}%
    \pgfmathsetmacro{\RobotBall}{0.15}%
    \pgfmathsetmacro{\RobotOC}{sqrt(\RobotSize*\RobotSize + \RobotRadius*\RobotRadius)}%
    \pgfmathsetmacro{\RobotAngleAOC}{atan(\RobotSize/\RobotRadius)}%
    \def\BallBeta{\RobotAngleAOC}
        
    
    \fill [black] (0,0) coordinate [label=left:$O$] (O) circle (1pt);
    \fill [black] (0:\RobotRadius) coordinate [label=below:$A_1$] (A1) circle (1pt);
    \fill [black] (\RobotAngle:\RobotRadius) coordinate [label=above:$A_2$] (A2) circle (1pt);
    \fill (A1) +(0,-\RobotSize) coordinate [label=below:$C_1$] (C1) circle (1pt);
    \fill[rotate=\RobotAngle] (A2) +(0,-\RobotSize) coordinate [label=above:$C_2$] (C2) circle (1pt);
    
    
    \draw[dotted] (O) -- (A1);
    \draw[dotted] (O) -- (A2);
    \draw[dotted] (O) -- (C1);
    
    % Ángulos
    \draw[dotted] (\RobotRadius/5,0) arc (0:\RobotAngle:\RobotRadius/5.0) node[pos=0.5,auto=right]{$\alpha$};
    \draw[dotted] (\RobotRadius/5,0) arc (0:-\RobotAngleAOC:\RobotRadius/5.0) node[pos=0.5,auto=left]{$\beta$};
    \draw[dotted] (A1) +(-\RobotRadius/5,0) arc (180:90+\RobotAngle/2:\RobotRadius/5.0) node[pos=0.5,auto=left]{$\gamma$};
    
    % Dibujar el robot en A1
    \draw[dotted] (A1) +(-\RobotSize/2,0) rectangle +(\RobotSize/2,-\RobotSize);
    \draw[dotted] (C1) circle(\RobotBall);
    
    %Dibujar el robot en A2
    \draw[dotted,rotate=\RobotAngle] (A2) +(-\RobotSize/2,0) rectangle +(\RobotSize/2,-\RobotSize);
    \draw[dotted] (C2) circle(\RobotBall);
    
   	%Trayectoria raton
   	%\pgfmathsetmacro{\XValueArc}{\ArcRadius*cos(\ArcAngle)}%
   	\draw[blue] (C1) arc (-\RobotAngleAOC:\RobotAngle-\RobotAngleAOC:\RobotOC) node[pos=0.5,auto=right]{$a$};
    
    %\draw (0,0) coordinate [label=right:$O$] (O) circle(4);
    \draw[dotted] (A1) arc (0:\RobotAngle:\RobotRadius);
    \draw[-latex,shorten >=1pt] (A1) -- (A2) node[pos=0.5,auto=right]{$\vec{r}$};
    
    %Dibujar el vector del raton
    \draw[red,-latex,shorten >=1pt] (C1) -- +(90-\RobotAngleAOC:\RobotRadius/3) node[pos=0.5,auto=right]{$\vec{m}$};
    
    
\end{tikzpicture}
\captionof{figure}{Diagrama de un giro del robot.\label{fig:giro}}
\end{center}

El vector $ \vec{r} $ indica el movimiento que realizó el robot en línea recta.
Conociendo el ángulo $\gamma$:

	$$ r_x ?= \overline{OA_1} \cdot e \cos(\gamma) $$
	$$ r_y ?= \overline{OA_1} \cdot \sin(\gamma) $$

\begin{center}
\begin{tikzpicture}[>=latex]
    \pgfmathsetmacro{\BallRadius}{3.0}%
    \pgfmathsetmacro{\BallOptoL}{2}%
    \pgfmathsetmacro{\BallOptoD}{0.5}%
    \pgfmathsetmacro{\BallBeta}{atan(\RobotSize/\RobotRadius)}%
    \pgfmathsetmacro{\BallBetaDouble}{90.0-\BallBeta*2.0}%
    
    \draw circle(\BallRadius);
    \fill [black] (0,0) coordinate [label=below:$O$] (O) circle (1pt);
    %Puntos de choque
    \fill [black] (0,-\BallRadius) coordinate [label=above:$B_x$] (Bx) circle (1pt);
    \fill [black] (-\BallRadius,0) coordinate [label=right:$B_y$] (By) circle (1pt);
    %Ejes
    \draw[dotted] (-\BallRadius,0) -- (\BallRadius,0);
    \draw[dotted] (0,\BallRadius) -- (0,-\BallRadius);
    %Cilindros de los ejes
    \draw[dotted] (By) +(\BallOptoD,-\BallOptoL) rectangle +(0,\BallOptoL);
    \draw[dotted] (Bx) +(-\BallOptoL,\BallOptoD) rectangle +(\BallOptoL,0);
    %Puntos
    \fill[black] (180-\BallBeta*2:\BallRadius) coordinate [label=above:$C_y$] (Cy) circle (1pt);
    \draw[blue] (By) -- (Cy) node[pos=0.5,auto=right]{$d_y$};
    \fill[black] (90-\BallBeta*2:\BallRadius) coordinate [label=above:$C_x$] (Cx) circle (1pt);
    \draw[blue] (Bx) -- (Cx) node[pos=0.5,auto=right]{$d_x$};
    %Ángulos
    %\draw[dotted] (Bx) +(-90:\BallRadius/5) arc (-90:-90-\BallBeta:\BallRadius/5.0) node[pos=0,auto=left]{$\beta$};
    %Dibujar el vector del raton
    \draw[red,-latex,shorten >=1pt] (O) -- +(90-\BallBeta:\BallRadius/2) node[pos=0.5,auto=right]{$\vec{m}$};

	\draw[dotted] (O) -- (Cx);
	%\draw[dotted] (Cx) +(270.0-\BallBeta:\BallRadius/5) arc (90.0-\BallBeta:\BallBetaDouble:\BallRadius/5.0) node[auto=right]{$\beta$};
	
	%\draw[dotted] (O) +(\BallBetaDouble:\BallRadius/4.0) arc (\BallBetaDouble:-90.0:\BallRadius/4.0) node[auto=right,pos=0.5]{$\delta$};
	
	\draw[dotted] (O) -- (Cy);
	%\draw[dotted] (O) +(90.0+\BallBetaDouble:\BallRadius/6.0) arc (90.0+\BallBetaDouble:180.0:\BallRadius/6.0) node[auto=above]{$\theta$};
	%node[auto=left,pos=0.5]{$\theta$}
    
    
    
    
    
    
\end{tikzpicture}
\captionof{figure}{Diagrama de la bola del ratón.\label{fig:bola}}
\end{center}

$$ \alpha = \widehat{OBC} $$

$$ d_y = R crd  $$

\section{Calibración de sensores}
\subsection{Ultrasonidos HC-SR04}

El sensor de ultrasonidos, permite averiguar de forma estimada la posición de un obstáculo. Para ello dispone de un emisor y un receptor de ultrasonidos colocados en la misma dirección, y posicionados en la parte delantera del robot.

Para calcular la distancia del objeto, se mide el tiempo que tarda el pulso ultrasónico en viajar desde el emisor al objeto, y tras rebotar, volver al receptor.

La velocidad del sonido, depende de la temperatura del aire $\vartheta$, además de otros factores\footnote{\url{http://en.wikipedia.org/wiki/Speed\_of\_sound\#Practical\_formula\_for\_dry\_air}}. Sin embargo, para una simplificación de los cálculos, se empleará la siguiente fórmula, asumiendo que es la misma en todas partes:

$$ v_{air} = 331.3 + 0.606\vartheta $$

Por lo tanto, si $t_1$ es el tiempo que tarda el sonido en llegar desde el emisor al obstáculo, y $t_2$ el tiempo desde el obstáculo al receptor:

$$ s_1 = v_{air} \cdot t_1 $$
$$ s_2 = v_{air} \cdot t_2 $$

Sin embargo, medimos $t$ que es $t_1+t_2$. Si suponemos que el robot esta parado cuando haga la medición, $s_1 = s_2$ y por tanto $t_1 = t_2$. Finalmente, $s$ es la distancia del sensor al obstáculo:

	$$ s = s_1 = s_2 = v_{air}\frac{t}{2} $$

Es importante mencionar que $s$ no indica la posición exacta del obstáculo, si no la distancia del mismo al sensor.

Para calibrar el sensor, es necesario determinar el ángulo de apertura $\theta_{sonar}$, que indica hasta que punto el sensor puede recibir el eco del sonido.

\begin{center}
\begin{tikzpicture}[>=latex]

	\pgfmathsetmacro{\SonarS}{6}
	\pgfmathsetmacro{\RobotSemiTheta}{\RobotThetaSonar/2.0}
    \pgfmathsetmacro{\RobotBall}{0.15}%
    \pgfmathsetmacro{\RobotOC}{sqrt(\RobotSize*\RobotSize + \RobotRadius*\RobotRadius)}%
    \pgfmathsetmacro{\RobotAngleAOC}{atan(\RobotSize/\RobotRadius)}%
    \def\BallBeta{\RobotAngleAOC}
        
    
    \fill [black] (0,0) coordinate [label=left:$O$] (O) circle (1pt);
    \fill [black] (-\RobotThetaSonar/2:\SonarS) coordinate [label=right:$A_1$] (A1) circle (1pt);
    \fill [black] (\RobotThetaSonar/2:\SonarS) coordinate [label=right:$A_2$] (A2) circle (1pt);
        \fill [black] (\RobotThetaSonar/4:\SonarS) coordinate [label=right:$C_1$] (C1) circle (1pt);
        \fill [black] (\RobotThetaSonar*3/4:\SonarS) coordinate [label=right:$C_2$] (C2) circle (1pt);
%    \fill (A1) +(0,-\RobotSize) coordinate [label=below:$C_1$] (C1) circle (1pt);
%    \fill[rotate=\RobotAngle] (A2) +(0,-\RobotSize) coordinate [label=above:$C_2$] (C2) circle (1pt);

	%(-\RobotThetaSonar/2:30:\SonarS) node[pos=0.5,auto=right]{$a$};
    
    
    \draw[dotted] (O) -- (A1);
    \draw[dotted] (O) -- (A2);
    %\draw[dotted] (O) -- (C1);

	%Dibujar anco del campo de visión
	\draw[dotted] (A2) arc[
		start angle=\RobotThetaSonar/2,
		delta angle=-\RobotThetaSonar,
		radius=\SonarS];
    
    % Ángulos
    %\draw[dotted] (\RobotThetaSonar/2:\SonarS/5) arc[
%		start angle=\RobotThetaSonar/2,
%		delta angle=-\RobotThetaSonar,
%		radius=\SonarS/5] node[auto]{$\theta_{sonar}$};
	%\pic ["$\theta$"] {angle = (A2)--(O)--(A1)};
	%\path[clip] (A1) -- (O) -- (A2);
	%\node[circle,draw=red,minimum size=3] at (0,0) (circ) {};

	%\coordinate (D) at (barycentric cs:A1=0.5 ,A2=0.5);
	%\node [above right] at (D) {D};
	\path[clip] (A1) -- (O) -- (A2) -- cycle;
	\draw [red, fill=red!20] (O) circle (30pt);
	%\draw [dotted] (O) circle (30pt)
	%\node[right] at (O) {holaa};


		
    
    % Ángulos
    %\draw[red] (\RobotRadius/5,0) arc (0:\RobotAngle:\RobotRadius/5.0) node[pos=0.5,auto=right]{$\theta_{sonar}$};
    %\draw[dotted] (\RobotRadius/5,0) arc (0:-\RobotAngleAOC:\RobotRadius/5.0) node[pos=0.5,auto=left]{$\beta$};
    %\draw[dotted] (A1) +(-\RobotRadius/5,0) arc (180:90+\RobotAngle/2:\RobotRadius/5.0) node[pos=0.5,auto=left]{$\gamma$};
    
    % Dibujar el robot en A1
    %\draw[dotted] (A1) +(-\RobotSize/2,0) rectangle +(\RobotSize/2,-\RobotSize);
    %\draw[dotted] (C1) circle(\RobotBall);
    
    %Dibujar el robot en A2
    %\draw[dotted,rotate=\RobotAngle] (A2) +(-\RobotSize/2,0) rectangle +(\RobotSize/2,-\RobotSize);
    %\draw[dotted] (C2) circle(\RobotBall);
    
   	%Trayectoria raton
   	%\pgfmathsetmacro{\XValueArc}{\ArcRadius*cos(\ArcAngle)}%
   	%\draw[blue] (C1) arc (-\RobotAngleAOC:\RobotAngle-\RobotAngleAOC:\RobotOC) node[pos=0.5,auto=right]{$a$};
    
    %\draw (0,0) coordinate [label=right:$O$] (O) circle(4);
    %\draw[dotted] (A1) arc (0:\RobotAngle:\RobotRadius);
    %\draw[-latex,shorten >=1pt] (A1) -- (A2) node[pos=0.5,auto=right]{$\vec{r}$};
    
    %Dibujar el vector del raton
    %\draw[red,-latex,shorten >=1pt] (C1) -- +(90-\RobotAngleAOC:\RobotRadius/3) node[pos=0.5,auto=right]{$\vec{m}$};
    
    
\end{tikzpicture}
\captionof{figure}{Sensor de ultrasonido.\label{fig:giro}}
\end{center}





\end{multicols}
\end{document}
